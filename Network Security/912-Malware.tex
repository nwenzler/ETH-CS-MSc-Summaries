\section{Malware Analysis and Prevention}

\subsection{Threat Analysis}

\paragraph{What to do with a Potential Threat} 
After finding out what to analyze, do automated testing to get a first impression - we may already know it. One can use internal sandboxes or use scanners. However, many malware samples try to prevent automated analysis (can detect a VM). Use a hex editor to find out if it's a portable executable (e.g. .exe) file. Such files have a distinct and well-defined structure.

Analysis generates a lot of metadata that one can further analyze and put into a database for clustering (e.g. does a website host a lot of malicious URLs?).

\paragraph{Blackboxing (Dynamic Analysis)}
Executing a sample on a dedicated system (virtual or real) to see what it does (not interested in how, we just wanna know what kind of malware it is). Monitor all system API calls, analyze logged information, dump decrypted content from memory (if malware is originally encrypted), etc.

Blackboxing can't tell you everything! Use Whiteboxing as well.

\paragraph{Whiteboxing (Static Analysis)}
Actually analyzing and reading the code by using a debugger or a disassembler. Can be helpful to find hidden functions (e.g. just doing malicious things on a specific day/condition, etc.).

\subsection{Prevention Methods}
A detection rate without knowing the false positive rate is useless. You could just classify everything as malicious to catch all malware (high false positive rate).

Also, see Firewall chapter.
%TODO: combine

\paragraph{Main Types}
\begin{itemize}
    \item \textbf{Pattern Matching:} signatures, static ML, data loss prevention, etc. - what is it?
    \item \textbf{Analysing Behavior:} anomaly detection, post-execution ML, web application firewall, etc. - what does it do?
    \item \textbf{Prevent Unwanted Access/Changes:} hardening, firewall, app isolation, etc.
\end{itemize}

\paragraph{Some Other Protection Concepts}

%TODO ???

\begin{itemize}
    \item \textbf{Deception:} honeypot an attacker.
    \item \textbf{User Entity Behavior Analysis:} zero trust.
    \item \textbf{Endpoint Detect and Response (EDR / XDR):} 
    \item \textbf{Network-Based Detection and ISP:} anomaly detection in network, etc.
\end{itemize}